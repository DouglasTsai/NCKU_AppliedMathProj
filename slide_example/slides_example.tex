\documentclass[11pt]{beamer}

\mode<article> % only for the article version
{
  \usepackage{fullpage}
  \usepackage{hyperref}
}
\usepackage{multimedia}
\usepackage{pgf,pgfarrows,pgfnodes,pgfautomata,pgfheaps,pgfshade}
\usepackage{amsmath,amssymb}
\usepackage[latin1]{inputenc}
%\usepackage{colortbl}
\usepackage[english]{babel}
%\usepackage{fancybox}
\usepackage{hyperref}
\usepackage{graphicx}


\mode<presentation> {
  \setbeamertemplate{background canvas}[vertical shading][bottom=white!10,top=blue!10]

  %\usetheme{Hannover} %...The third theme
  %\usetheme{Montpellier} %...The second theme
  \usetheme[hideothersubsections]{Marburg} %....The first theme
  \usefonttheme[onlymath]{serif}
  \usecolortheme{sidebartab}
  \pgfdeclareimage[height=1.5cm]{logo}{1111021021} \logo{\pgfuseimage{logo}}
  \pgfdeclareimage[height=6cm]{NCKUSC}{NCKUSC} \logo{\pgfuseimage{NCKUSC}}
}

%\setbeamercolor{math text}{fg=green!50!black}
%\setbeamercolor{normal text in math text}{parent=math text}


%\usepackage{lmodern}
%\usepackage[T1]{fontenc}

\usepackage{times}

\setbeamercovered{dynamic}


%\useoutertheme{split}
%\useinnertheme[shadow]{rounded}
\title{Henon-Heiles 模型研究計劃}
\subtitle{科學計算導論簡報範例}
\author{組員A, 組員B,組員C,組員D,組員E}

\institute{Dept. of Math, NCKU, Tainan}
\date{May 25, 2009}
%\nodate
%---- BEGIN NEW COMMANDS
\newcommand{\vc}[1]{\mbox{\boldmath$#1$\unboldmath}}
\newcommand{\bmathsf}[1]{\vc{\mathsf{#1}}}
\newcommand{\intd}{\int_{\mbox{\footnotesize \sf D}}}
\newcommand{\inti}{\int_{\mbox{\footnotesize\sf I}}}
\newcommand{\intih}{\int_{\hat{\mbox{\footnotesize\sf I}}}}
\newcommand{\intbd}{\oint_{\mbox{\footnotesize ${\delta}$\sf D}}}
\newcommand{\pfpx}[2]{\frac{\partial #1}{\partial #2}}
\newcommand{\dfdx}[2]{\frac{d #1}{d #2}}
\newcommand{\ldnorm}[1]{\| #1 \|^2_{\mbox{\footnotesize \sf D}} }
\newcommand{\spc}[1]{\mbox{\sf #1}}
\newcommand{\ope}[1]{\cal #1}
\newcommand{\mt}[1]{{\rm #1}}
\newcommand{\innerd}[2]{\left( #1,#2 \right)_{\mbox{\footnotesize \sf D}}}
\newcommand{\inners}[2]{\left( #1,#2 \right)_{\mbox{\footnotesize ${\delta}$\sf D}}}
\newcommand{\summ}[2]{\displaystyle \sum_{i=0}^{#1}\sum_{j=0}^{#2}}
\newcommand{\sqrend}{\begin{flushright}$\square$\end{flushright}}
\newcommand{\pfend}{\begin{flushright}$\square$\end{flushright}}
\newcommand{\pf}{\emph{Proof. }}
\newcommand{\R}[2]{R_#1^{(#2)}}
\renewcommand{\baselinestretch}{1}
%---- END NEW COMMANDS
\begin{document}

\frame{\titlepage}
\begin{frame}
  \frametitle{Outline}
  \tableofcontents
  % You might wish to add the option [pausesections]
\end{frame}

\section{計劃目的}

\begin{frame}{計劃目的}
\begin{itemize}
\item 動機:

\item 計劃目的:

本計劃目的在研究Henon-Heiles模型解的行為隨著參數及初始條件變動的情形.

\end{itemize}
\end{frame}
\section{計劃內容}
\begin{frame}{Model Equations}

Henon-Heiles模型如下
\begin{align*}
\frac{d}{dt}y_1&=y_2\\
\frac{d}{dt}y_2&=-y_1-2 y_1 y_3\\
\frac{d}{dt}y_3&=y_4\\
\frac{d}{dt}y_4&=-y_3-y_1^2+y_3^2
\end{align*}
\end{frame}


\begin{frame}{Analysis}

由Henon-Heiles模型可得以下方程
$$\frac{1}{2}(y_1^2+y_2^2+y_3^2+y_4^2)+y_1^2y_3-\frac{1}{3} y_3^3=h$$
數值模擬程序:
\begin{enumerate}
\item 先決定$h$值, 考慮適當$y_1$,$y_3$,$y_4$初始值,
再由上式得到$y_2$初始值.
\item 以ode45得到前一頁Henon-Heiles模型數值解
\item 選取恰當的Poincare section得到Poincare map
\end{enumerate}
\end{frame}
\section{初步結果}
\begin{frame}{$h=0.13$, $y_1=y_3=0$, $y_4=0.215$}

\includegraphics[width=\textwidth]{./scicomp/islands2.png}
\end{frame}

\begin{frame}{$h=0.13$, $y_1=y_3=0$, $y_4=0.225$}

\includegraphics[width=\textwidth]{./scicomp/circles.png}

\end{frame}
\begin{frame}{$h=0.13$, $y_1=y_3=0$, $y_4=0.18$}

\includegraphics[width=\textwidth]{./scicomp/chaos.png}

\end{frame}

\section{計劃規劃}
\begin{frame}{計劃規劃}
\begin{itemize}
\item 規劃:
\begin{enumerate}
\item
\end{enumerate}
\item 任務分配:
\begin{itemize}
\item 資料搜集: 組員A
\item 理論推導: 組員C
\item 程式撰寫: 組員D
\item 成果整理: 組員B, 組員E
\end{itemize}
\end{itemize}

\end{frame}

\section{參考資料}
\begin{frame}{參考資料}
\begin{itemize}
\item Computer modeling: from sports to spaceflight ... from order to
chaos. by Danby, J. M. A.
\end{itemize}

\end{frame}
\end{document}
